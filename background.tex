\chapter{\normalfont BACKGROUND}
Advances in sequencing technology have generated an enormous amount of data over the past decade \parencite{van2014ten}. Equally advanced computational methods are needed to conduct comparative and functional genomic studies on these datasets. These methods include the application of a substitution (chapter 2) model and the development of an insertion-deletion (chapter 4) model of evolution for phylogenetic inference, multiple sequence alignment, and ancestral state reconstruction, as well as post-alignment fixation of coding sequences prevalent in genomic datasets \parencite{ranwez2011macse}, through indel-phase profiling method (chapter 3).  

Substitution models of evolution are used to make predictions about the substitution process in molecular sequences along the branches of a tree or phylogeny. These models can be classified into two main groups: empirical models, which are based on observed frequencies of substitutions in sets of alignments, and mechanistic models, which use estimated evolutionary parameters to calculate substitution matrices. These parameterized models are more advanced in that they can take into account how mutational biases and natural selection may skew these observed transition rates \parencite{yang2008mutation}. Parameter optimization for these models are commonly done through a Maximum Likelihood (ML) method. In the second chapter, I present an application of an expectation-maximization algorithm for maximum-likelihood training of a codon substitution model (MG94+GTR). 

Insertion-deletion (Indels) mutations are one of the most common classes of mutations in the genome, second only to single base substitution \parencite{taylor2004occurrence}. Indels are normally generated by DNA polymerase errors or incorrect DNA repair during the replication process. The evolutionary history of indels is complex and often involves repetitive genomic regions, which makes identification, alignment, and annotation difficult \parencite{kunkel2004dna}. Therefore, accurate and reproducible methods of identifying indels are crucial for understanding their evolutionary patterns and predicting unknown gene functions. In the third chapter, I present an application of a post-alignment fixation method (sliding-window method) to profile each indel event into three different phases (Figure 3.1) according to its position within a codon. This method helps to find all three indel phase events that improves upon current codon-aware aligners that can only support phase 0 indels. 

Compared to substitution models, indel models are far less developed due to their complexity. For example, indel rates are normally inferred based on gap counts, while a single gap can reflect more than one event \parencite{miklos2004long}. The excess of deletion over insertion events (\cite{ophir1997patterns, zhang2003patterns, gu1995size}), also requires models to evaluate separate indel length distributions. In the fourth chapter, I present an innovative indel-phase model that makes several improvements to current models: 1) while previous models for indel evolution assumes one indel rate, our model proposes three rates for each insertion and deletion event based on the corresponding indel phases; 2) this model is richer because of the inclusion of an MG94 + GTR substitution model within; 3) I apply the importance sampling method within the EM algorithm to remove the alignment bias when inferring parameters from the indel-phase model. The simulation results demonstrate the efficiency and accuracy of my indel-phase model. 

 In the last chapter, I extend our indel phase analysis to a published 90 pairwise species across the tree of life, including 15 eukaryotic, 6 archaeal, and 69 bacterial clades. Also, I apply three different alignment methods on this dataset and compare the phase proportions and distributions between them. I find the effective purifying selection existing withing coding regions by looking at the phase proportions across 90 species pairs.  

In summary, I develop and evaluate methods for profiling the indel phases and estimating the evolutionary parameters derived from substitution and indel models in pairwise alignments. The post-alignment fixation method (sliding-window) is able to generate better alignments, where better alignments help improve the phylogenetic inference, ancestral sequence reconstruction, and gene annotation area \parencite{rosenberg2009sequence}. Besides, the EM algorithm and importance sampling method is capable of estimating the indel rates of three different phases and length distributions from genomic data by applying my innovative indel-phase model. There are more ways to do so, and many more that are not discussed here. Each has strengths and weaknesses, but understanding what these are is critical to success. 

