\chapter{\normalfont CONCLUSION}
While insertions and deletions (indels) are common molecular evolutionary events, the study of indel phases and probabilistic models for indel evolution have received much less attention due to their model complexity. In this work, I developed and evaluated methods for profiling the indel phases and estimating the evolutionary parameters derived from substitution and indel models in pairwise alignments. In chapter 2 I provided an example of the expectation maximization algorithm for maximum-likelihood training for a substitution model, where the structural context of a residue was treated as a hidden variable that can evolve over time. In chapter 3 I developed a post-alignment fixation method that was capable of profiling each indel event into three different phases where the current codon-aware aligner was unable to. Expanding on the model established in chapters 2 and 3, I developed an indel-phase model that could describe the substitution and indel process together in pairwise sequences. In chapter 5 I extended our indel phase analysis to a more complex dataset via three different alignment methods, which utilized some important results from all previous chapters. 

Current genomic datasets are largely unpolished and require advanced models and algorithms to handle uncertainties in alignments, erroneous estimates of evolutionary parameters, and other issues that could negatively impact comparative and functional genomic studies. Our substitution and indel models of evolution can be incorporated into future alignment software to robustly model coding sequence evolution and improve upon current aligners that do not support indel phases. 
 
