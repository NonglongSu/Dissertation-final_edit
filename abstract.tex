  \begin{abstract}
    Advances in sequencing technology have generated an enormous amount of data over the past decade. Equally advanced computational methods are needed to conduct comparative and functional genomic studies on these datasets, in particular tools that appropriately interpret indels within an evolutionary framework. The evolutionary history of indels is complex and often involves repetitive genomic regions, which makes identification, alignment, and annotation difficult. While previous studies have found that indel lengths in both deoxyribonucleic acid and proteins obey a power law, probabilistic models for indel evolution have rarely been explored due to their computational complexity. \\ 
    \indent In my research, I first explore an application of an expectation-maximization algorithm for maximum-likelihood training of a codon substitution model. Next I compare the training accuracy of the substitution model with a derivative-free optimization method -- Nelder-Mead Algorithm. Then I apply the expectation-maximization algorithm and substitution model on a published 90 pairwise species dataset and analyze the obtained parameter estimates. \\
    \indent Second, I develop a post-alignment fixation method (sliding-window method) to profile each indel event into three different phases according to its codon position. Because current codon-aware models are only able to identify the indels by placing the gaps between codons and can lead to the misalignment of the sequences. I find that the focus species (mouse-rat) are under purifying selection by looking at the differences among three indel phase proportions. I also demonstrate the power of the sliding window method by comparing the post-aligned and original gap positions. \\
    \indent Third, I create an indel-phase probabilistic model including the indel rates of different phases and length distributions, where the substitution model is included so that the model can describe the indel and substitution process together. Then I establish a Gillespie simulation to generate a series of true sequence alignments. I develop an importance sampling method within the expectation-maximization algorithm to train the model and infer evolutionary parameters from the alignments.  \\ 
    \indent Finally, I extend the indel phase analysis to the 90 pairwise species data from chapter 2 across three alignment methods, including Mafft+sw method developed in chapter 3, coati-sampling methods applied in chapter 4, and coati-max method. Also I explore the relationship between the dN/dS and Zn/(Zn+Zs) ratio across 90 species pairs. 
  \end{abstract}
    
  